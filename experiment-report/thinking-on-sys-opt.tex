\section{对系统优化工作的感想}

计算机系统是一个有限的系统,对其进行优化就是找到各种权衡(trade off)点,
放弃一些我们不在乎的特性,来换取一些我们关注的性能。

\subsection{软件结构和新算法}

本人目前接触到的所有优化相关的知识和技巧,
都是在讨论如何提升软件在并发条件下的性能,
主要是从软件结构入手进行优化。
比如FrozenHot的无锁不可变前端缓存\cite{qiu_frozenhot_2023},
线程池用的WorkStealing队列,
本人认为都是一种软件结构上的创新。
它们都是为了减少线程间不必要的互操作,
但由于其应用场景不同需求不同,其实现形式差异很大。
这说明计算机系统方向的优化,虽有成千上万的应用场景,但存在若干基本的共有的权衡点,
优化工作可以从这些共同的点入手。

要了解定位到这些点并不容易,需要从业人员对计算机通用硬件、底层软件的原理有深刻的理解和把握。
看似简单的计数器,开发者可以在多个线程间简单地原子访问,
也可以根据需求不同,设计使用引用计数、设定上限的近似计数
\footnote{软件对计数器当前具体数据不感兴趣,只对计数总量接近设定值时感兴趣}
等计数器
\cite{perfbook}。
前者可对其原子访问内存序进行优化:增加引用量不会导致生命期管理事件,任何线程对其大小均不感兴趣,因此可采用relaxed内存序;
引用计数减小可能会导致内存回收,所有线程需要了解精确的计数大小, 同时也希望其他线程即使看见自己对计数的修改,
因此采用acq\_rel内存序。
后者各线程局部维护一个计数器,各线程在计数器未接近指定数值之前线程间减少共享数据的频率,
来减少竞争,提升CPU缓存一致性。
上述优化基本是深度结合应用需求和计算机软硬件底层原理,
在软件结构上进行优化得来的。

在缓存优化工作领域,上述方法为优化缓存并发读写性能。
除此之外,还有很多工作结合机器学习等新算法对缓存淘汰算法进行优化,
考核文章中的 GL-Cache 就提出通过分组学习的方式,减少各元素平均学习成本\cite{yang_gl-cache_nodate}。
不仅从对学习算法进行了创新,还从软件结构层面减小了学习引入的读写开销,
是一种综合两种方向的优化。

无论是优化算法,还是优化读写,本质上都是为提高缓存吞吐量而服务的。
因此优化工不能拘泥于一个方向,应像GL-Cache一样,结合发挥多种想法的优势为优化的根本目标服务。

\subsection{新硬件}

根据本人的了解,新硬件主要为了解决通用硬件难以满足某些特定需求而设计的。
计算方面如专为AI计算优化的各种APU,NPU,TPU etc. 以及相对更灵活的FPGA.
存储方面有持久内存,更廉价的固态硬盘等。
新硬件的加入不会改变计算机系统是有限系统的事实,
新硬件只是拓展了其解决特定问题的能力,因此上文观点仍然适用。

不过现有软件是为通用硬件设计的,如果不对软件加以修改,很难发挥新硬件的优势。
比如NBJL的FlatLSM,对RocksDB做了适应PMEM的改造,
PMEM-Table实现了通用硬件下WAL 和 MEM-Table两个组件的功能。
通过对LSM结构的改造,免除WAL,减少$L_0, L_1$的写放大问题,
写性能有显著提升\cite{he_flatlsm_2023}。

新硬件不但拓展了计算机解决特定问题的能力,
也为优化工作带来新的机会和突破口。
需要开发人员继续结合应用需要以及新硬件特性进行探索, 寻找新的性能权衡点。

\subsection{新应用场景}

随着5G和云计算的发展,目前越来越多的应用上云,这不仅是出于经济的层面考虑,
结合低延迟通信和密集的计算存储资源可以创造出新的应用场景,
比如云游戏, 中心化的区域智能交通规划等。

上述场景,针对系统优化工作来说,个人认为主要是以应用集群化,分布式化作为大的优化方向。
除了能够提供弹性横向并发规模拓展,计算和存储集中化也是集群带来的优势。
集群内较低的通信成本可以允许应用在集群中做一些计算或存储资源共享。
比如 CaaS-LSM 微服务化 Compaction,使得LSM更适合集群场景\cite{qiaolin_yu_caas-lsm_2024}。

\subsection{实验后感受}

优化工作需要大量的尝试和思考。
本次实验中实现的FIFO-Hybrid-LRU方案经过几次修改和迭代,甚至推倒重做。
试错的过程也是人学习的过程,改的越多,人学到的就越多。
LRU的思想也许一页纸就能概括,但完成整个实验所习得的经验不是一页纸就能概括的了的,
这种感觉是只看书得不到的。

除了计算机基础知识,数学和问题建模能力也是非常重要的。
一些参数化问题往往能够利用数学建模的手段直接解决,机器学习理论也是基于数学发展而来的。
在看了Kangaroo文后的数学建模后,深感本人在数学方面还有巨大的进步空间。
