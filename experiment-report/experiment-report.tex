\maketitle

\section{实验报告}

\subsection{技术选型}

本次实验主要实现 FrozenHot\cite{qiu_frozenhot_2023} 的动态缓存部分。
采用 LRU 置换算法,本时延中的该置换算法基于链表和hash表作为数据结构。
hash 函数选用 MurmurHash3 \cite{SMHasher}

本次考核文献\cite{qiu_frozenhot_2023}采用一个不可变无锁前端缓存,显著减少了缓存在并发管理方面的开销。
文中讨论了 lock contention 对系统性能的负面影响。出于相同的目的,本实验采用lock-free链表。
并在\cite{timothy_l_harris_pragmatic_2001}方案的基础上,增加\verb|splice|操作,
以满足置换算法的需要。

\subsection{实现细节}


\subsection{性能测试}

\subsubsection{测试方法}

本实验分别对
\begin{enumerate}
    \item 单线程缓存(基准)
    \item 多线程条件下一把锁保护的基准缓存
    \item 为多线程设计实现的缓存
\end{enumerate}
进行测试,按照实验要求测试数据量为$10^6$项,每项key为8B,value为16B 的数据。
其中第二项和第三项将会测试其在不同线程数量时的性能对比。
上述三种缓存实现均采用LRU作为淘汰算法,
因此只在基准测试中采用多种随机生成分布,
在对上述第2项和第3项测试时,仅采用 Zipf 分布\cite{dirtyzipf2024}。

\subsubsection{测试环境}

\begin{itemize}
    \item Intel Core I7 8750H 2.20GHz - 4.10GHz
    \item 16GB RAM
    \item \verb|GCC 13.3.1 20240614|
    \item \verb|Linux 6.6.47-gentoo-dist x86_64 GNU/Linux|
\end{itemize}
